\section[Intro]{Introduction}

\begin{frame}{}
  \begin{center}
    \structure{\Large \insertsection}
  \end{center}
\end{frame}

\begin{frame}{Common Problems in Transports \& Logistics}

  \begin{columns}[b]
  \column{150pt}
    Common Problems
    \begin{itemize}
      \item<2-> efficient supply chains
      \item<3-> process/job scheduling
      \item<4-> tour designing
      \item<5-> resource planing
      \item<5-> ...
    \end{itemize}
  \column{129pt}
%   \includegraphics<2>[height=3cm]{images/logistics_cargo.jpg}
%    \includegraphics<3>[height=3cm]{images/robot_arm.jpg}
%    \includegraphics<4>[height=3cm]{images/TSP_Deutschland_3.png}
%    \includegraphics<5>[height=3cm]{images/BatteryCup.png}
  \end{columns}
  \uncover<6->{
    \vspace{5pt}
    Common Goal:
    \begin{itemize}
      \item profit maximization
      \item cost minimization
      \item ecological, financial, environmental, social, ...
    \end{itemize}
  }
\note{
It is all about Optimization of some value - whatever its financial, social, environmental, ...}
\end{frame}

\begin{frame}{Optimization Problems}
  ``... finding the \emph{best} solution from all \emph{feasible} solutions'' 
  \hfill  {\tiny (Wikipedia) }\\

\uncover<2->{
  \vspace{5pt}
  Ways to find the \emph{best} solution
  \begin{itemize}
    \item exact methods
      \begin{itemize}
        \item<3-> (non)linear programming, ...
        \item<3-> usually expensive
	\item<3-> sometimes impossible
	\item<4-> LVA 186835 Mathematical Programming VU 3.0
      \end{itemize}
    \item heuristic methods
      \begin{itemize}
        \item<5-> usually cheaper
	\item<5-> higher accuracy is costly
	\item<6-> LVA 186112 Heuristic Optimization Techniques VU 3.0
      \end{itemize}
  \end{itemize}
  \uncover<7->{
    \vspace{5pt}
    \alert{Problem:} Not all information is available in advance!
  }

}
\end{frame}

\begin{frame}{Offline vs. Online Problems }

  \begin{block}{Definition: Offline}
    In an \alert{offline} problem, all data is known in advance.
  \end{block}
  \begin{block}{Definition: Online}
    In an \alert{online} or \alert{dynamic} problem, some data is known in advance, while the remaining data is revealed as time passes.
  \end{block}


  \begin{center}
    \only<1>{\vspace{2.55cm}}

%    \includegraphics<2>[height=3cm]{images/Truck_Stop.jpg}
%    \includegraphics<3>[height=3cm]{images/Ambulance_Graz_side2.jpg}
%
%    \includegraphics<4>[height=3cm]{images/On-Off_Switch.jpg}
%    \includegraphics<5>[height=3cm]{images/assembly_line.jpg}
%
%    \includegraphics<6>[height=3cm]{images/screen_switch.jpg}
%    \includegraphics<6>[height=3cm]{images/server_switch.jpg}
  \end{center}

  \only<7>{
    \begin{itemize}
      \item application specific strategies
      \item solutions will be worse
    \end{itemize}
    \vspace{1.05cm}
  }

  \note{
    How shall we react on incoming requests while the system is running?

    Maybe stop cargo trucks along the way, but can't stop emergency ambulance

    Dont want to switch off your assembly lane or computing infrastructure

    Answer to this is problem/application specific

    The question is HOW bad the solutions will be!
  }

\end{frame}

\begin{frame}{Comparing Offline vs. Online Algorithms}
  \begin{block}{Definition: Competitive Analysis}
    An online algorithm $A$ is \alert{$r$-competitive} if there exists a constant $\alpha$, such that its solution $C_A$ on any instance of the problem is no more than $r$ times the solution $C_{OPT}$ of an offline algorithm plus $\alpha$. \\
    
    \vspace{5pt}
    
    \begin{center}
      $C_A(I) \le r * C_{OPT}(I) + \alpha, \forall problem\ instances\ I$
    \end{center}

    \vspace{5pt}
  \end{block}

  \begin{itemize}
    \item<2-> popular and widely used
    \item<3-> comparing worst cases
    \item<4-> no performance measure
  \end{itemize}
\end{frame}

\begin{frame}{Comparing Offline vs. Online Algorithms}
  other methods:
  \begin{itemize}
    \item Max/Max Ratio
    \item Random Order Ratio
    \item Bijective Analysis
    \item Average Analysis
    \item Relative Worst Order Analysis
    \item problem specific methods
  \end{itemize}

  \vspace{10pt}

  \begin{itemize}
    \item different methods favor different algorithms
    \item hard to compare!
  \end{itemize}
\end{frame}

\begin{frame}{Dynamism}
  \begin{block}{Definition: Degree of Dynamism}
    The \alert{Degree of Dynamism} describes how much the system changes dynamically during execution.

    \begin{center}
      $dod = \frac{n_{dyn}}{n_{total}}$\\
      $0 \le dod \le 1$
    \end{center}

  \end{block}

  \begin{itemize}
    \item<2-> might be insufficient
    \item<2-> does not consider request time
  \end{itemize}

  \begin{block}<3->{Definition: Effective Degree of Dynamism}
    The \alert{Effective Degree of Dynamism} describes how late during execution dynamic changes are applied to the system.

    \begin{center}
      $edod = \frac{\sum_{i=1}^n(t_i/T) }{n}$\\
      $0 \le edod \le 1$
    \end{center}

  \end{block}

\end{frame} 
