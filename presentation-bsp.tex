\documentclass{beamer}

\usepackage[utf8x]{inputenc}
\usepackage{default}
\usepackage[german]{babel}
\usepackage{pgfpages}

\usetheme{Frankfurt}
\usecolortheme{default}
\useinnertheme{rounded}
% | default | inmargin |
%	rectangles | rounded
%}

\setbeamercovered{transparent}
\setbeamertemplate{footline}[frame number]
\setbeameroption{show notes on second screen=right}
%\setbeameroption{hide notes}

\title{An Introduction to Online Algorithms in Transports and Logistics}

\author[P. Schmidt]{
	Peter Schmidt\\ 
	MNr. 0526202
}

\begin{document}

\frame{
  \titlepage
}
\frame{
  \frametitle{Schedule}
  \tableofcontents
  \note{
    Short Introduction on Optimization Problems in TO\\
    Present some terminology for the domain\\
    Define what on-/offline means\\
    Show some basic principles using examples\\
    ask right away - discussion afterwards
  }
}

\section[Intro]{Introduction}

\begin{frame}{}
  \begin{center}
    \structure{\Large \insertsection}
  \end{center}
\end{frame}

\begin{frame}{Common Problems in Transports \& Logistics}

  \begin{columns}[b]
  \column{150pt}
    Common Problems
    \begin{itemize}
      \item<2-> efficient supply chains
      \item<3-> process/job scheduling
      \item<4-> tour designing
      \item<5-> resource planing
      \item<5-> ...
    \end{itemize}
  \column{129pt}
   \includegraphics<2>[height=3cm]{images/logistics_cargo.jpg}
    \includegraphics<3>[height=3cm]{images/robot_arm.jpg}
    \includegraphics<4>[height=3cm]{images/TSP_Deutschland_3.png}
    \includegraphics<5>[height=3cm]{images/BatteryCup.png}
  \end{columns}
  \uncover<6->{
    \vspace{5pt}
    Common Goal:
    \begin{itemize}
      \item profit maximization
      \item cost minimization
      \item ecological, financial, environmental, social, ...
    \end{itemize}
  }
\note{
It is all about Optimization of some value - whatever its financial, social, environmental, ...}
\end{frame}

\begin{frame}{Optimization Problems}
  ``... finding the \emph{best} solution from all \emph{feasible} solutions'' 
  \hfill  {\tiny (Wikipedia) }\\

\uncover<2->{
  \vspace{5pt}
  Ways to find the \emph{best} solution
  \begin{itemize}
    \item exact methods
      \begin{itemize}
        \item<3-> (non)linear programming, ...
        \item<3-> usually expensive
	\item<3-> sometimes impossible
	\item<4-> LVA 186835 Mathematical Programming VU 3.0
      \end{itemize}
    \item heuristic methods
      \begin{itemize}
        \item<5-> usually cheaper
	\item<5-> higher accuracy is costly
	\item<6-> LVA 186112 Heuristic Optimization Techniques VU 3.0
      \end{itemize}
  \end{itemize}
  \uncover<7->{
    \vspace{5pt}
    \alert{Problem:} Not all information is available in advance!
  }

}
\end{frame}

\begin{frame}{Offline vs. Online Problems }

  \begin{block}{Definition: Offline}
    In an \alert{offline} problem, all data is known in advance.
  \end{block}
  \begin{block}{Definition: Online}
    In an \alert{online} or \alert{dynamic} problem, some data is known in advance, while the remaining data is revealed as time passes.
  \end{block}


  \begin{center}
    \only<1>{\vspace{2.55cm}}

    \includegraphics<2>[height=3cm]{images/Truck_Stop.jpg}
    \includegraphics<3>[height=3cm]{images/Ambulance_Graz_side2.jpg}

    \includegraphics<4>[height=3cm]{images/On-Off_Switch.jpg}
    \includegraphics<5>[height=3cm]{images/assembly_line.jpg}

    \includegraphics<6>[height=3cm]{images/screen_switch.jpg}
    \includegraphics<6>[height=3cm]{images/server_switch.jpg}
  \end{center}

  \only<7>{
    \begin{itemize}
      \item application specific strategies
      \item solutions will be worse
    \end{itemize}
    \vspace{1.05cm}
  }

  \note{
    How shall we react on incoming requests while the system is running?

    Maybe stop cargo trucks along the way, but can't stop emergency ambulance

    Dont want to switch off your assembly lane or computing infrastructure

    Answer to this is problem/application specific

    The question is HOW bad the solutions will be!
  }

\end{frame}

\begin{frame}{Comparing Offline vs. Online Algorithms}
  \begin{block}{Definition: Competitive Analysis}
    An online algorithm $A$ is \alert{$r$-competitive} if there exists a constant $\alpha$, such that its solution $C_A$ on any instance of the problem is no more than $r$ times the solution $C_{OPT}$ of an offline algorithm plus $\alpha$. \\
    
    \vspace{5pt}
    
    \begin{center}
      $C_A(I) \le r * C_{OPT}(I) + \alpha, \forall problem\ instances\ I$
    \end{center}

    \vspace{5pt}
  \end{block}

  \begin{itemize}
    \item<2-> popular and widely used
    \item<3-> comparing worst cases
    \item<4-> no performance measure
  \end{itemize}
\end{frame}

\begin{frame}{Comparing Offline vs. Online Algorithms}
  other methods:
  \begin{itemize}
    \item Max/Max Ratio
    \item Random Order Ratio
    \item Bijective Analysis
    \item Average Analysis
    \item Relative Worst Order Analysis
    \item problem specific methods
  \end{itemize}

  \vspace{10pt}

  \begin{itemize}
    \item different methods favor different algorithms
    \item hard to compare!
  \end{itemize}
\end{frame}

\begin{frame}{Dynamism}
  \begin{block}{Definition: Degree of Dynamism}
    The \alert{Degree of Dynamism} describes how much the system changes dynamically during execution.

    \begin{center}
      $dod = \frac{n_{dyn}}{n_{total}}$\\
      $0 \le dod \le 1$
    \end{center}

  \end{block}

  \begin{itemize}
    \item<2-> might be insufficient
    \item<2-> does not consider request time
  \end{itemize}

  \begin{block}<3->{Definition: Effective Degree of Dynamism}
    The \alert{Effective Degree of Dynamism} describes how late during execution dynamic changes are applied to the system.

    \begin{center}
      $edod = \frac{\sum_{i=1}^n(t_i/T) }{n}$\\
      $0 \le edod \le 1$
    \end{center}

  \end{block}

\end{frame}



\section[TSP]{Travelling Salesman Problem}

\begin{frame}{}
  \begin{center}
    \structure{\Large \insertsection}
  \end{center}
\end{frame}


\begin{frame}{Travelling Salesman Problem}
  ``Given a list of cities and the distances between each pair of cities, what is the shortest possible route that visits each city exactly once and returns to the origin city?'' \hfill  {\tiny (Wikipedia) }

\vspace{5pt}
  \begin{columns}[T]
  \column[T]{93pt}
    \includegraphics<2->[width=90pt]{images/djibouti_anim1.png}
  \column[T]{93pt}
    \includegraphics<3->[height=100pt]{images/djibouti_data.png}
    \only<-2>{\vspace{100pt}}
  \column[T]{93pt}
    \includegraphics<4>[width=90pt]{images/djibouti_anim2.png}
    \includegraphics<5->[width=90pt]{images/djibouti_anim3.png}
  \end{columns}

\vspace{5pt}
\uncover<6>{\uncover<6>{Possibly the most studied combinatorial optimization problem}}

\end{frame}

\begin{frame}{Offline TSP}
  \begin{itemize}
    \item<2-> ``pure'' TSP can be ``easily'' solved exactly
    \item<3-> biggest instance solved (so far)
      \begin{columns}[T]
      \column[T]{120pt}
	\begin{itemize}
	  \item 85,900 ``cities''
	  \item solved in 2006 \\after 15 years ``race''
	  \item approx. 136 CPU years \\{\tiny scaled to 2.4 GHz AMD Opteron}
	  \item<5> Djibouti: $\sim$ 0.2 s
	\end{itemize}
	\vspace{50pt}
      \column[T]{140pt}
	\includegraphics<4->[width=140pt]{images/pla8tour_small.png}
      \end{columns}
  \end{itemize}
\end{frame}

\begin{frame}{Online TSP}
  \begin{itemize}
    \item TSP: new requests arise while ``on the road''
    \item adds \emph{time} as a new factor
    \item not comparable with ``pure'' TSP
    \item corresponding offline problem: \emph{TSP with release dates}
  \end{itemize}
\end{frame}

\begin{frame}{TSP-RD}
  \begin{block}{Definition: TSP-RD}
    In a \alert{TSP with Release Dates}, find a round-trip tour stating at the origin, s.t. all cities are visited exactly once \emph{after their release date}.
  \end{block}
  Preliminaries:
  \begin{itemize}
    \item $n$ Cities; $N=\{1, ... ,n\}; i \in N $
    \item Cities have a location ($l_i$) in a metric space.
    \item Cities are revealed at their disclosure date ($q_i \ge 0$).
    \item Cities are ready for service at their release date ($r_i \ge q_i$).
    \item The service time at each city is 0.
    \item The salesman travels at constant speed (1 unit of distance per unit of time) or is idle.
    \item The problem begins at time 0 and the salesman is idle at the origin ($i=0$).
  \end{itemize}
\end{frame}

\begin{frame}{TSP-RD on $\Bbb{R}^+$}
  Consider cities with location in $\Bbb{R}^+$ (non-negative real line)

  \vspace{5pt}
  \begin{block}{Optimal Offline Algorithm}
    \begin{enumerate}
      \item Go directly to to city $l_{max} = max_{i \in N}\{l_i\}$.
      \item Wait at city $l_{max}$ for $max_{i \in N}\{max\{0,r_i - 2l_{max}+l_i\}\}$.
      \item Proceed directly back to the origin, visiting the cities.
    \end{enumerate}
  \end{block}

  \begin{block}{Remark}
    The offline algorithm can ignore the disclosure dates $q_i$, because it knows the data in advance!
  \end{block}

  $C_{OPT}(n) = 2l_{max} + max_{i \in N}\{max\{0,r_i - 2l_{max}+l_i\}\}$
  \note{blackboard demo}
\end{frame}

\begin{frame}{Online TSP-RD on $\Bbb{R}^+$}
  Online algorithms
  \begin{itemize}
    \item need to consider disclosure dates $q_i$
    \item don't know final value of $n$
  \end{itemize}
  \vspace{5pt}

  \begin{block}<2->{Move-Right-If-Necessary Algorithm}
    consider $q_i=r_i$
    \begin{enumerate}
      \item If there is an unserved city to the right, move toward it.
      \item If there are no unserved cities to the right, move toward the origin.
      \item Become idle at the origin.
    \end{enumerate}
  \end{block}

  \begin{Theorem}<3->
    $C_{MRIN}(n) \le \frac {3}{2} C_{OPT}(n), \forall n$.
  \end{Theorem}

  \note{blackboard demo}
\end{frame}

\begin{frame}{Online TSP-RD on $\Bbb{R}^+$}

  \begin{block}{Move-Left-If-Beneficial Algorithm}
    consider $q_i \le r_i$
    \begin{enumerate}
      \item If there is an unserved city to the right, move toward it.
      \item If there are no unserved cities to the right, move toward the origin \alert{iff} we can visit all unserved cities on/after their release date; otherwise stay idle at the current position.
      \item Become idle at the origin.
    \end{enumerate}
  \end{block}

  \begin{block}{Remark}
    If $q_i = r_i$, MLIB is indistinguishable from MRIN!\\
    If $q_i = 0$, MLIB is indistinguishable from the offline algorithm!
  \end{block}

  \note{blackboard demo}
\end{frame}

\begin{frame}{Online TSP-RD on generic metric space}
  Additions to the problem:
  \begin{itemize}
    \item locations in a generic metric space (e.g., Euclidean Plane)
    \item distance function $d(\cdotp,\cdotp)$, 
    \item Origin $o$
    \item availability of a ``black box'' algorithm solving offline TSP-RD
  \end{itemize}
  \vspace{5pt}

  \begin{block}{Plan-At-Home-with-disclosure-dates Algorithm}
    \begin{enumerate}
      \item Whenever at origin, follow a tour that serves all cities whose disclosure dates have passed.
      \item If at time $q_i$ the server is at position $p$ and a new request arrives at location $x$, then
      \begin{itemize}
        \item if $d(x,o) > d(p,o)$, go back to origin, continue with (1).
        \item if $d(x,o) \le d(p,o)$, ignore it.
      \end{itemize}
    \end{enumerate}
  \end{block}
\end{frame}

\begin{frame}{PAH-dd}
  \begin{Theorem}
    Algorithm PAH-dd is 2-competitive
  \end{Theorem}
  \begin{Proof}
    part. proof (for case 1);\\
    $T$ is optimal tour length at time $q_n$ from ``black box'' alg.
    \begin{small}
    \begin{align}
      C_{PAH-dd} & =  q_n + T \\
		 &\le r_n + T \\
		 &\le C_{OPT} + T \\
		 &\le C_{OPT} + C_{OPT} \\
		 &\le 2* C_{OPT}
    \end{align}
    \end{small}
  \end{Proof}
\end{frame}

\begin{frame}{PAH-dd}
  \begin{Proof}
    part. proof (for case 2a);\\
    $d(o,l_n)<d(o,p(q_n))$ \\
    implies: arrive at $o$ before $q_n + d(o,l_n)$
    \begin{small}
    \begin{align}
      C_{PAH-dd} &  < q_n + d(o,l_n) + T \\
		 &\le r_n + d(o,l_n) + T \\
		 &\le C_{OPT} + T \\
		 &\le C_{OPT} + C_{OPT} \\
		 &\le 2* C_{OPT}
    \end{align}
    \end{small}
  \end{Proof}
\end{frame}

\begin{frame}{PAH-dd}
  \begin{Proof}
    part. proof (for case 2b) omitted
  \end{Proof}

  \begin{block}{Remark}
    By assuming, that $q_i = r_i - a$, and $\alpha = \frac{a}{L_{TSP}}$, it can be shown, that PAH-dd is $(2-\frac{\alpha}{1+\alpha})$-competitive.\\
    (This resolves to $a/L_{TSP}+a)$, thus with $a \sim L_{TSP}$ PAH-dd is $\frac{3}{2}$-competitive.
  \end{block}
\end{frame}

\begin{frame}{TSP Competitiveness Examples}
  For TSP without disclosure-dates $q_i \le r_i$
  \begin{itemize}
    \item on metric space: 2-competitive
    \item polynomial on metric space: 2.78-competitive
    \item on $\Bbb{R}$: 1.64-competitive
    \item on $\Bbb{R}^+$: 1.5-competitive
  \end{itemize}
\end{frame}


\section[VRP]{Vehicle Routing Problem}

\begin{frame}{}
  \begin{center}
    \structure{\Large \insertsection}
  \end{center}
\end{frame}

\begin{frame}{Vehicle Routing Problem}

  ``... service a number of customers with a fleet of vehicles.'' 
  \hfill  {\tiny (Wikipedia) }\\
  \vspace{5pt}
  \begin{itemize}
    \item e.g., delivering goods from a central depot
    \item vehicle routes must be planned for delivery
  \end{itemize}

  \begin{center}
    \only<1>{\vspace{2.55cm}}
    \includegraphics<2>[height=3cm]{images/VRP_setting.png}
    \includegraphics<3->[height=3cm]{images/VRP_solution.png}
  \end{center}

  \uncover<4>{
    \begin{itemize}
      \item<4> multi objective optimization:
	\begin{itemize}
	  \item optimize number of tours (reduce vehicles)
	  \item optimize tour length (reduce variable costs)
	\end{itemize}
    \end{itemize}
  }

\end{frame}

\begin{frame}{Vehicle Routing Problem}

  \begin{itemize}
    \item major problem in logistics \& transportation
    \item extensions by additional constraints
    \item<2-> common extensions: \\Vehicle Routing Problem \alert{with Time Windows and Capacity Constraints}
      \begin{itemize}
        \item customers can only be served within specified time windows
	\item customers demand specific quantity
	\item vehicles have (the same) maximum capacity
      \end{itemize}
    \item<3> online problem: 
      \begin{itemize}
        \item new requests arrive during execution of tours
	\item update existing tours
	\item create new tour
      \end{itemize}
  \end{itemize}
   
\end{frame}

\begin{frame}{Online-VRP-TWCC}
  Considering a practical implementation using Ant Colony Systems

  \begin{block}{Ant Colony Optimization}
    \alert{Ant Colony Optimization} is a nature inspired heuristic mimicking the behaviour of ants.
  \end{block}
  \begin{itemize}
    \item Ants spread pheromones while travelling from food to nest
    \item Short paths are travelled more often, leading to higher pheromone concentration
    \item Ants follow pheromone traces more likely than other routes
  \end{itemize}

\end{frame}

\begin{frame}{Ant Colony Systems}
  \begin{itemize}
    \item Ants decisions based on 
    \begin{itemize}
      \item global information (pheromone traces)
      \item local information (ant knows its current neighbourhood)
    \end{itemize}
    \item decisions are probabilistic, not deterministic
    \item start with initial solution (by some construction heuristic)
    \item improve solution stepwise until termination
    \item termination after iterations/time
  \end{itemize}

  \begin{Definition}
    $\tau_{ij}(t)$\dots global pheromone value on arc $(i,j)$ at time $t$ \\ 
    $\eta_{ij}$\dots local attractiveness (e.g., inverse distance)
  \end{Definition}

  \note{
    TAU \\
    ETA
  }
\end{frame}

\begin{frame}{Ant Colony Systems}
  Parameters:
  \begin{itemize}
    \item $q_0 \in [0,1]$\dots biasing towards intensification or diversification
    \item $\beta$\dots control influence of local information
  \end{itemize}
  \vspace{5pt}
  \uncover<2>{
    Deciding, which way to go \dots
    \begin{itemize}
      \item for every neighbour $j \in N$, calculate value $x_j = \tau_{ij}*(\eta_{ij})^\beta$
      \item choose random $q \in [0,1]$
      \begin{itemize}
	\item if $q \le q_0$: choose $l$, s.t. $x_l = max(x_j)$ (intensification)
	\item if $q > q_0$: choose $l$ randomly with $p_j = \frac{x_j}{\sum_l(x_l)}$ (diversification)
      \end{itemize}
    \end{itemize}
  }
\end{frame}

\begin{frame}{Ant Colony Systems}
  Parameters:
  \begin{itemize}
    \item $\tau_0$\dots initial pheromone value
    \item $\rho \in [0,1]$\dots evaporation factor
  \end{itemize}
  \vspace{5pt}
  \uncover<2>{
    Modifying pheromone values \dots
    \begin{itemize}
      \item local updates: applied directly after an ant traverses an arc\\
	    $\tau_{ij}(t)=(1-\rho)\tau_{ij}(t) + \rho*\tau_0$
      \item global updates: applied after iteration is evaluated\\
	    $\tau_{ij}(t)=(1-\rho)\tau_{ij}(t) + \rho*\Delta$\\
	    where $\Delta$ is derived from the current best solution.
    \end{itemize}
  }
\end{frame}

\begin{frame}{Solving DVRP-TWCC with ACS}
  How to solve the DVRP-TWCC:
  \begin{itemize}
    \item multi objective optimization with multiple ACS
    \item \emph{ACS-VEI}: reduce number of vehicles
    \item \emph{ACS-TIME}: optimize tours found by ACS-VEI
    \item use independent pheromone trails
    \item share current best solution $\psi^{gb}$
  \end{itemize}

  \note{
    PSI
  }
\end{frame}

\begin{frame}{Solving DVRP-TWCC with ACS}
  Basic algorithm:
  \begin{itemize}
    \item initialize $\psi^{gb}$ with a construction heuristic
    \item try to improve it using both ACS:
      \begin{itemize}
        \item ACS-VEI: find feasible solution with one vehicle less than $\psi^{gb}$
	\item ACS-TIME: optimize tour length of $\psi^{gb}$ using the same vehicles
      \end{itemize}
    \item update $\psi^{gb}$ on success
    \item if number of vehicles is reduced, restart ACSs
  \end{itemize}
\end{frame}

\begin{frame}{Solving DVRP-TWCC with ACS}
  Artificial Ants:
  \begin{itemize}
    \item ants start at depot
    \item move to unvisited node, considering TW and CC
    \item attractiveness $\eta_{ij}$ by
      \begin{itemize}
        \item travelling time
	\item time windows
	\item number of times, node $j$ has not been visited (for ACS-VEI)
      \end{itemize}
    \item ant-solution might miss customers
      \begin{itemize}
        \item sort by decreasing demands
	\item insert by shortest travel time
	\item considering constraints
      \end{itemize}
  \end{itemize}
\end{frame}

\begin{frame}{Solving DVRP-TWCC with ACS}
  Pheromone Values:
  \begin{itemize}
    \item $J^h_{\psi}$ \dots the length of initial solution by construction heuristic
    \item $J^{gb}_{\psi}$ \dots the length of current best solution
    \item $n$ \dots number of nodes
    \item minimum pheromone values: $\tau_0 = \frac{1}{n*J^h_{\psi}}$
    \item local updates: $\tau_{ij} = (1-\rho)*\tau_{ij} + \rho*\tau_0$\\
    \item global update: $\tau_{ij} = (1-\rho)*\tau_{ij} + \frac{\rho}{n*J^{gb}_{\psi}}$\\	  
  \end{itemize}
\end{frame}

\begin{frame}{Solving DVRP-TWCC with ACS}
  Results:
  \begin{itemize}
    \item based on classic benchmarks
    \item compare MACS-DVRP-TWCC with MACS-VRP-TWCC
    \item Parameters: \\
	  $n=100$ nodes\\
	  $m=10$ ants\\
	  $q_0=0.9$ (focus intensification)\\
	  $\beta = 1$\\
	  $\rho = 0.1$\\
    
  \end{itemize}
\end{frame}
\begin{frame}{Results}
  \begin{center}

  uniformly random customers\\
  \begin{tabular}{|c|c|c|c|}
    \hline
    algorithm	& veh.	& dist.	& r.c.	\\
    \hline
    MACS-DVRP	& 14	& 1428	& 9	\\
    MACS-DVRP	& 14	& 1508	& 12	\\
    MACS-DVRP	& 15	& 1367	& 14	\\
    \hline
    MACS-VRP	& 13	& 1214	& x	\\
    \hline
  \end{tabular}
  
  \vspace{15pt}

  clustered customers\\
  \begin{tabular}{|c|c|c|c|}
    \hline
    algorithm	& veh.	& dist.	& r.c.	\\
    \hline
    MACS-DVRP	& 11	& 1218	& 9	\\
    MACS-DVRP	& 12	& 1124	& 12	\\
    MACS-DVRP	& 11	& 1278	& 14	\\
    \hline
    MACS-VRP	& 10	& 828	& x	\\
    \hline
  \end{tabular}
  \end{center}

\end{frame}

\begin{frame}{MACS-DVRP-TWCC Conclusion}

  MACS conclusion:
  \begin{itemize}
    \item not really online solution
    \item basically a start-stop recalculation of multiple static VRPs
    \item works, because ACS usually fast
    \item ACS typically good for TSP
  \end{itemize}

\uncover<2>{
  extension:
  \begin{itemize}
    \item divide time horizon $T$ (a day) in periods
    \item accumulate requests, recalculate at end of period
    \item solve problem as VRP with heterogenous starting points and capacities (according to current position and remaining capacity)
  \end{itemize}
}
\end{frame}


\begin{frame}{Alternative Approaches}
  Multi Agent Auctioning
  \begin{itemize}
    \item distributed system of agents (vehicles)
    \item on request, calculate costs of insertion
    \item best offer ``wins''
    \item self-organisation allows exension (real-time traffic info, etc)
  \end{itemize}

  Comparing Alternatives:
  \begin{itemize}
    \item is hard/impossible
    \item focus different aspects
    \item test results based on different data sets
    \item non-deterministic behaviour
    \item too many parameters
  \end{itemize}

\end{frame}

\section[Conclusion]{Conclusion}

\begin{frame}{}
  \begin{center}
    \structure{\Large \insertsection}
  \end{center}
\end{frame}

\begin{frame}{Conclusion}
  Online Algorithms in Transportation and Logistics
  \begin{itemize}
    \item young discipline
    \item active research
    \item importance of dynamic/online problems in real world applications
    \item comparing approaches is hard
    \item highly problem specific
  \end{itemize}
\end{frame}

\begin{frame}{Literature}
  Algorithms taken from
  \begin{itemize}
    \item Jaillet, P. and M. Wagner. ``Online Routing Problems: Value of Advanced Information as Improved Competitive Ratios''. Transportation Science, 40, 200--210 (2006).
    \item Ahmmed, Ashek et al. ``A Multiple Ant Colony System for Dynamic Vehicle Routing Problem with Time Window''. Proceedings of the 2008 Third International Conference on Convergence and Hybrid Information Technology - Volume 02
  \end{itemize}
\end{frame}

\begin{frame}{}
  \begin{center}
    \structure{\Large Thank you for your attention!}
  \end{center}
\end{frame}


\end{document}