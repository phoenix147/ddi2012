\section[VRP]{Vehicle Routing Problem}

\begin{frame}{}
  \begin{center}
    \structure{\Large \insertsection}
  \end{center}
\end{frame}

\begin{frame}{Vehicle Routing Problem}

  ``... service a number of customers with a fleet of vehicles.'' 
  \hfill  {\tiny (Wikipedia) }\\
  \vspace{5pt}
  \begin{itemize}
    \item e.g., delivering goods from a central depot
    \item vehicle routes must be planned for delivery
  \end{itemize}

  \begin{center}
    \only<1>{\vspace{2.55cm}}
%    \includegraphics<2>[height=3cm]{images/VRP_setting.png}
%    \includegraphics<3->[height=3cm]{images/VRP_solution.png}
  \end{center}

  \uncover<4>{
    \begin{itemize}
      \item<4> multi objective optimization:
	\begin{itemize}
	  \item optimize number of tours (reduce vehicles)
	  \item optimize tour length (reduce variable costs)
	\end{itemize}
    \end{itemize}
  }

\end{frame}

\begin{frame}{Vehicle Routing Problem}

  \begin{itemize}
    \item major problem in logistics \& transportation
    \item extensions by additional constraints
    \item<2-> common extensions: \\Vehicle Routing Problem \alert{with Time Windows and Capacity Constraints}
      \begin{itemize}
        \item customers can only be served within specified time windows
	\item customers demand specific quantity
	\item vehicles have (the same) maximum capacity
      \end{itemize}
    \item<3> online problem: 
      \begin{itemize}
        \item new requests arrive during execution of tours
	\item update existing tours
	\item create new tour
      \end{itemize}
  \end{itemize}
   
\end{frame}

\begin{frame}{Online-VRP-TWCC}
  Considering a practical implementation using Ant Colony Systems

  \begin{block}{Ant Colony Optimization}
    \alert{Ant Colony Optimization} is a nature inspired heuristic mimicking the behaviour of ants.
  \end{block}
  \begin{itemize}
    \item Ants spread pheromones while travelling from food to nest
    \item Short paths are travelled more often, leading to higher pheromone concentration
    \item Ants follow pheromone traces more likely than other routes
  \end{itemize}

\end{frame}

\begin{frame}{Ant Colony Systems}
  \begin{itemize}
    \item Ants decisions based on 
    \begin{itemize}
      \item global information (pheromone traces)
      \item local information (ant knows its current neighbourhood)
    \end{itemize}
    \item decisions are probabilistic, not deterministic
    \item start with initial solution (by some construction heuristic)
    \item improve solution stepwise until termination
    \item termination after iterations/time
  \end{itemize}

  \begin{Definition}
    $\tau_{ij}(t)$\dots global pheromone value on arc $(i,j)$ at time $t$ \\ 
    $\eta_{ij}$\dots local attractiveness (e.g., inverse distance)
  \end{Definition}

  \note{
    TAU \\
    ETA
  }
\end{frame}

\begin{frame}{Ant Colony Systems}
  Parameters:
  \begin{itemize}
    \item $q_0 \in [0,1]$\dots biasing towards intensification or diversification
    \item $\beta$\dots control influence of local information
  \end{itemize}
  \vspace{5pt}
  \uncover<2>{
    Deciding, which way to go \dots
    \begin{itemize}
      \item for every neighbour $j \in N$, calculate value $x_j = \tau_{ij}*(\eta_{ij})^\beta$
      \item choose random $q \in [0,1]$
      \begin{itemize}
	\item if $q \le q_0$: choose $l$, s.t. $x_l = max(x_j)$ (intensification)
	\item if $q > q_0$: choose $l$ randomly with $p_j = \frac{x_j}{\sum_l(x_l)}$ (diversification)
      \end{itemize}
    \end{itemize}
  }
\end{frame}

\begin{frame}{Ant Colony Systems}
  Parameters:
  \begin{itemize}
    \item $\tau_0$\dots initial pheromone value
    \item $\rho \in [0,1]$\dots evaporation factor
  \end{itemize}
  \vspace{5pt}
  \uncover<2>{
    Modifying pheromone values \dots
    \begin{itemize}
      \item local updates: applied directly after an ant traverses an arc\\
	    $\tau_{ij}(t)=(1-\rho)\tau_{ij}(t) + \rho*\tau_0$
      \item global updates: applied after iteration is evaluated\\
	    $\tau_{ij}(t)=(1-\rho)\tau_{ij}(t) + \rho*\Delta$\\
	    where $\Delta$ is derived from the current best solution.
    \end{itemize}
  }
\end{frame}

\begin{frame}{Solving DVRP-TWCC with ACS}
  How to solve the DVRP-TWCC:
  \begin{itemize}
    \item multi objective optimization with multiple ACS
    \item \emph{ACS-VEI}: reduce number of vehicles
    \item \emph{ACS-TIME}: optimize tours found by ACS-VEI
    \item use independent pheromone trails
    \item share current best solution $\psi^{gb}$
  \end{itemize}

  \note{
    PSI
  }
\end{frame}

\begin{frame}{Solving DVRP-TWCC with ACS}
  Basic algorithm:
  \begin{itemize}
    \item initialize $\psi^{gb}$ with a construction heuristic
    \item try to improve it using both ACS:
      \begin{itemize}
        \item ACS-VEI: find feasible solution with one vehicle less than $\psi^{gb}$
	\item ACS-TIME: optimize tour length of $\psi^{gb}$ using the same vehicles
      \end{itemize}
    \item update $\psi^{gb}$ on success
    \item if number of vehicles is reduced, restart ACSs
  \end{itemize}
\end{frame}

\begin{frame}{Solving DVRP-TWCC with ACS}
  Artificial Ants:
  \begin{itemize}
    \item ants start at depot
    \item move to unvisited node, considering TW and CC
    \item attractiveness $\eta_{ij}$ by
      \begin{itemize}
        \item travelling time
	\item time windows
	\item number of times, node $j$ has not been visited (for ACS-VEI)
      \end{itemize}
    \item ant-solution might miss customers
      \begin{itemize}
        \item sort by decreasing demands
	\item insert by shortest travel time
	\item considering constraints
      \end{itemize}
  \end{itemize}
\end{frame}

\begin{frame}{Solving DVRP-TWCC with ACS}
  Pheromone Values:
  \begin{itemize}
    \item $J^h_{\psi}$ \dots the length of initial solution by construction heuristic
    \item $J^{gb}_{\psi}$ \dots the length of current best solution
    \item $n$ \dots number of nodes
    \item minimum pheromone values: $\tau_0 = \frac{1}{n*J^h_{\psi}}$
    \item local updates: $\tau_{ij} = (1-\rho)*\tau_{ij} + \rho*\tau_0$\\
    \item global update: $\tau_{ij} = (1-\rho)*\tau_{ij} + \frac{\rho}{n*J^{gb}_{\psi}}$\\	  
  \end{itemize}
\end{frame}

\begin{frame}{Solving DVRP-TWCC with ACS}
  Results:
  \begin{itemize}
    \item based on classic benchmarks
    \item compare MACS-DVRP-TWCC with MACS-VRP-TWCC
    \item Parameters: \\
	  $n=100$ nodes\\
	  $m=10$ ants\\
	  $q_0=0.9$ (focus intensification)\\
	  $\beta = 1$\\
	  $\rho = 0.1$\\
    
  \end{itemize}
\end{frame}
\begin{frame}{Results}
  \begin{center}

  uniformly random customers\\
  \begin{tabular}{|c|c|c|c|}
    \hline
    algorithm	& veh.	& dist.	& r.c.	\\
    \hline
    MACS-DVRP	& 14	& 1428	& 9	\\
    MACS-DVRP	& 14	& 1508	& 12	\\
    MACS-DVRP	& 15	& 1367	& 14	\\
    \hline
    MACS-VRP	& 13	& 1214	& x	\\
    \hline
  \end{tabular}
  
  \vspace{15pt}

  clustered customers\\
  \begin{tabular}{|c|c|c|c|}
    \hline
    algorithm	& veh.	& dist.	& r.c.	\\
    \hline
    MACS-DVRP	& 11	& 1218	& 9	\\
    MACS-DVRP	& 12	& 1124	& 12	\\
    MACS-DVRP	& 11	& 1278	& 14	\\
    \hline
    MACS-VRP	& 10	& 828	& x	\\
    \hline
  \end{tabular}
  \end{center}

\end{frame}

\begin{frame}{MACS-DVRP-TWCC Conclusion}

  MACS conclusion:
  \begin{itemize}
    \item not really online solution
    \item basically a start-stop recalculation of multiple static VRPs
    \item works, because ACS usually fast
    \item ACS typically good for TSP
  \end{itemize}

\uncover<2>{
  extension:
  \begin{itemize}
    \item divide time horizon $T$ (a day) in periods
    \item accumulate requests, recalculate at end of period
    \item solve problem as VRP with heterogenous starting points and capacities (according to current position and remaining capacity)
  \end{itemize}
}
\end{frame} 
